\documentclass{article}

\usepackage{amssymb,amsmath,longtable, gensymb}

\usepackage[pdftex]{graphicx}
\graphicspath{ {Images/} {Images/eLoss/} {Images/CoordSys/} {Images/TrackerPics/} {Images/Field/} {Images/TrackingFlow/}}

\usepackage[margin=1in]{geometry}

\usepackage{hyperref}
\hypersetup{
    colorlinks=true,
    linkcolor=red,
    urlcolor=blue,
    citecolor=blue
}

\usepackage[backend=biber,sorting=none]{biblatex}
\addbibresource{mybib.bib}



\begin{document}

\title{Geane Track Fitting}
\author{Nicholas Kinnaird}
\date{\today}
\maketitle

% \renewcommand{\abstractname}{Assigned Problem}
\begin{abstract}

    In this document I will detail the concepts and mathematics of GEANE track fitting, as well as it's implementation in the gm2 simulation framework for the Muon g-2 Experiment at Fermilab. I will also detail the pecularities and intricacies of the code. This is done both for documentation purposes as well as a precursor to sections of my future thesis. This document assumes some familiarity with art on part of the reader, though such knowledge is not necessary for many sections. Note that the tracking code and framework is in constant development, so the details contained here within may be outdated at time of reading. 

\end{abstract}



\section{Introduction}

  The Muon g-2 Experiment at Fermilab will use tracking detectors in order to measure positron trajectories for the purpose of determining the beam distribution and its characteristics, both for the final $\omega_{a}$ result and for general beam diagnostics. A fraction of positrons originating from decaying muons will pass through the trackers. By fitting these tracks and extrapolating back to the average decay point, the beam can be characerized in a non-destructive fashion. The tracking can be done independently from the calorimeters in order to cross-check, or combined with them in order to provide the muon distribution that the calorimeters see for the direct $\omega_{a}$ calculation. It is arguable that this is the most important duty of the trackers. (Note that the integration of the two detector systems in the code has not been initiated.) It is also worth noting that there is a large percentage of tracks that hit the calorimeters hit zero or a small number of tracking modules, which this fitting code is not capable of handling.

  Each tracker module consists of 4 layers of 32 straws with a stereo angle of 7.5 degrees, the first two ``U'' layers oriented with the tops of the straws at a greater radial position, and the second two ``V'' layers oriented with the bottoms of the straws at a greater radial position. A tracking module is shown in Figure \ref{fig:tracker}. There are 3 tracker stations located at the 0, 12th, and 18th sections of the ring, counting clockwise from the top most point of the ring where the inflector resides. (Station 18 has been installed at time of writing, with station 0 planned for the fall. Station 12 is slated for the future sometime.) Each station consists of 8 tracking modules arranged in a staircase pattern that follows the curvature of the ring as seen in Figure \ref{fig:staircase}. Further hardware and electronics information regarding the trackers will be omitted in this document.

\begin{figure}[]
\caption{Shown is a picture of one of the many tracking modules used in the Muon g-2 experiment. The first layer of straws with a stereo angle of 7.5 degrees can be seen, with the other 3 straw layers hiding behind it. The beam direction is roughly into the page in this picture, and this view is what the decay positrons will see. Picture provided by James.}
\centering
\includegraphics[width=0.9\textwidth]{Tracker}
\label{fig:tracker}
\end{figure}

\begin{figure}[]
\caption{Tracker modules are arranged in the shown staircase pattern. In green and dark blue is the edge of the vacuum chamber (where the dark blue identifies the modification that was made to the old vacuum chambers), and it can be seen that vacuum chamber walls lie at the ends of the outside tracking modules. The position of a calorimeter can be seen in teal at the right. The dark red spots are the locations of the pole tips.}
\centering
\includegraphics[width=0.9\textwidth]{trackerStation}
\label{fig:staircase}
\end{figure}


  Because of the proximity of the trackers to the muon beam, they will lie within a region of varying field. The radial field of the trackers rises from 0 Tesla at the outer ends to roughly .3 Tesla at the inner top and bottom ends, and the vertical field drops approximately 50\% from the storage dipole field of 1.451 Tesla. Shown in Figures \ref{fig:operaBy} and \ref{fig:operaBx} is the location of the tracker with respect to the horizontal and vertical fields respectively. If one thing can be said to be special to g-2 when it comes to tracking it is these large field gradients over the tracking detector region and several meter long extrapolation distance back to the muon decay point. This is one of the main motivations for using the Geane (Geometry and Error Propagation) fitting algorithm and routines, which has direct access to the field. 


\begin{figure}[]
\caption{Shown is the vertical field of the g-2 magnet in and around the storage region as calculated in Opera 2D. The center of the storage region lies at 7.112 m along the x axis. The black box shows the rough location of the tracker with respect to the field (size exaggerated slightly). It can be seen that there is a large inhomogeneity within the tracker space, goring from left to right.}
\centering
\includegraphics[width=0.9\textwidth]{operaBy}
\label{fig:operaBy}
\end{figure}

\begin{figure}[]
\caption{Shown is the radial field of the g-2 magnet in and around the storage region as calculated in Opera 2D. The center of the storage region lies at 7.112 m along the x axis. The black box shows the rough location of the tracker with respect to the field (size exaggerated slightly). It can be seen that there is a large homogeneity at the inner upper and lower ends compared to the right center. The shape of the pole pieces and tips can readily be seen.}
\centering
\includegraphics[width=0.9\textwidth]{operaBx}
\label{fig:operaBx}
\end{figure}

  The Geane fitting routines originated in Fortran with the EMC collaboration, and was used in the precursor E821 experiment as well as the PANDA experiment with some success. (cite stuff here?) (There might be some other instances of its use as well.) The core error propagation routines were at some point added to Geant4 under the error\_propagation directory which is included in all default installs. The tracking code strengths lie with its direct implementation and access to the Geant4 geometry and field, and its ability to handle the field inhomogeneties. The Geane algorithm code which makes use of the Geant4 error\_propagation routines follows the structure of \cite{geanemanual} and is detailed in the Formalism section in this paper. It is a relatively straight forward least squares global $\chi^{2}$ minimization algorithm. 



\section{Formalism}
\label{sec:Formalism}

    The author recommends reading the references \cite{geanemanual}, \cite{trajfit}, and \cite{Lavezzi} in order to best understand the fitting algorithm. However, due to the at times confusing notation, ommitted equations or concepts, and differences between papers, I have attempted to summarize here the different sources and present the material in a more understandable and readable format, while at the same time drawing connections to the implemented code.

    One can define a $\chi^{2}$ for a track in the normal way by dividing the residuals of measured and predicted track parameters by their errors,
        \begin{align} \label{eq:chi2}
            \chi^2 = (\vec{p}-\vec{x})^{T} (\sigma^{-1}) (\vec{p}-\vec{x}),
        \end{align}
    where $\vec{p}$ are predicted track parameters from a fit to the measured track parameters $\vec{x}$, and $\sigma$ is a matrix of errors on the fitted parameters.

    The Geant4 error\_propagation routines can be used to determine these predicted parameters and error matrices by propagating track parameters from some initial guesses. By minimizing this $\chi^{2}$ with respect to the track parameters one can then fit and improve the track. The Geant4 routines propagate particles along their average trajectories neglecting the effects of discrete processes, using a helix equation along small enough steps where the change in the magnetic field is small. The parameters are then a function of path length, 
        \begin{align} \label{eq:pp}
            p_{l} = F_{l,l_{0}}(p_{0}),
        \end{align}
    where the coordinate system transforms described above can be used to determine parameters at specific X values, or detector planes. (From here on the dependence on path length or X position will be neglected, in favor of using plane indices.) In tandem, error matrices describing the expected distribution in true parameters about those predicted parameters due to said discrete process are also calculated:
        \begin{align} \label{eq:sigma}
            \sigma^{ij} = <p^{i}p^{j}> - <p^{i}> \cdot <p^{j}>,
        \end{align} 
    where i and j are track parameter indices.

    The propagation of these parameters and error matrices are done using transport matrices, which express the infinitesimal changes in parameters at some plane (or path length) with respect to the parameters at some previous plane (or previous path length):
        \begin{align} \label{eq:transport}
            \delta p_{N} = T_{N,N-1} \delta p_{N-1}, \\
            \sigma_{N} = T_{N,N-1} \sigma_{N-1} T_{N,N-1}^{T}.
        \end{align}
 
    Said transport and error matrices are 5x5 objects since the parameter vectors are 5x1 objects as described above. With parameters defined on such planes, one can define the $\chi^{2}$ as: 
        \begin{align} \label{eq:chi2sum}
            \chi^2 = \sum_{i=1}^{N} [(p_{i}(p)-x_{i})^{T} (\sigma_{i}^{-1}) (p_{i}(p)-x_{i})],
        \end{align}
    where $p_{i}$ are the average predicted parameters from some general starting parameters $p$. At first order one can only include the measurement error on parameters, which fill in the diagonals of $\sigma_{i}$, if random processes can be neglected. It is important to note that unmeasured parameters should have measurement errors of ininity (or some large value) along the diagonals in the code, which save the fact that residuals for unmeasured parameters do not exist. When the error matrix is inverted all rows and columns of the matrix with these large numbers will fall to 0 in the $\chi^{2}$. 

    In order to get the best fit track, the $\chi^{2}$ should be minimized with respect to the initial track parameters p, and evaluated at some chosen parameters:
        \begin{align} \label{eq:minimize}
            \frac{\partial \chi^{2}}{\partial p}|_{p=p'_{0}} = 0,
        \end{align}
    resulting in
        \begin{align}
            0 = \sum_{i=1}^{N}[ (\frac{\partial p_{i}(p)}{\partial p}|_{p=p'_{0}})^{T} (\sigma_{i}^{-1}) (p_{i}(p'_{0})-x_{i}) \\ 
            + (p_{i}(p'_{0})-x_{i})^{T} \frac{\partial(\sigma_{i}^{-1})}{\partial p}|_{p=p'_{0}} (p_{i}(p'_{0})-x_{i}) \\ 
            +  (p_{i}(p'_{0})-x_{i})^{T} (\sigma_{i}^{-1}) (\frac{\partial p_{i}(p)}{\partial p}|_{p=p'_{0}})]
        \end{align}
    where the 1st and 3rd terms are identical, and the 2nd term is small if one assumes that error matrix doesn't change much with respect to the starting parameters. (Fair since most of the error comes from measurement, and as long as the initial guess is decent enough such that the path length through material doesn't change much from one iteration to the next.)

    This simplifies to: 
        \begin{align} \label{eq:solve}
            \sum_{i=1}^{N} T^{T}_{i0} (\sigma_{i}^{-1}) (p_{i}(p'_{0})-x_{i}) = 0.
        \end{align}
    To solve this simply make the substitution 
        \begin{align} \label{eq:psub}
            p_{i}(p'_{0}) = p_{i}(p_{0}) + \frac{\partial p_{i}(p_{0})}{\partial p} \Delta p_{0} = p_{i}(p_{0}) + T_{i0} \Delta p_{0},
        \end{align}
    where $p'_{0}$ are the improved starting parameters for the next iteration calculated from the previous starting parameters $p_{0}$, and $\Delta p_{0}$ are the changes in the starting parameters to improve the track. This equation can be plugged into the above if one makes the assumption that $\frac{\partial p_{i}(p)}{\partial p} = T_{i0}$ does not change much from one iteration the next, which follows from the inherent nature of making small adjustments to the track in order to improve it.

    Thus one arrives at 
        \begin{align} \label{eq:deltap}
            \Delta p_{0} = \sigma_{p_{0}} \sum_{i=1}^{N} T^{T}_{i0}(\sigma_{i}^{-1})(x_{i} - p_{i}(p_{0})),
        \end{align}
    where
        \begin{align} \label{eq:cov}
            \sigma_{p_{0}} = [\sum_{i=1}^{N} T^{T}_{i0} (\sigma_{i}^{-1}) T_{i0} ]^{-1},
        \end{align}
    is the 5x5 covariance matrix of fitted parameters on the starting plane, whose diagonals describe the errors in the 5 track parameters on that plane and in the region close to it. (The fit does not directly return fit errors for track parameters on other planes.) 

    However, since random processes should not be neglected for optimal tracking results, it makes more sense to return to the original $\chi^2$ in equation \ref{eq:chi2}, only now the included matrix and vector objects are combined into one large linear algebra equation. Instead of a sum over N 5x1 objects multiplying 5x5 error matrices, the vectors are combined into a single 5Nx1 vector multiplying a single 5Nx5N matrix. The 5x5 diagonals of this large error matrix should now include the effects due to material processes as calculated in Geant from equation \ref{eq:sigma} as well as the measurement errors. 

    Because now parameters at one plane are no longer independent of the parameters at other planes, due to correlations from these random processes, it's necessary to add off-diagonal elements into the large error matrix. These 5x5 blocks come from 
        \begin{align} \label{eq:corr}
            \sigma_{MN} = T_{MN} \sigma_{N}, 
        \end{align}
    for the top diagonals, and the transpose for the bottom diagaonals, where M and N are two separate planes within the detector. ($\sigma_{N}$ is the error matrix on plane N calculated from the starting plane.) This follows from equation \ref{eq:sigma} evaluated at plane M with respect to a path length from plane N, which is equivalent to \ref{eq:corr}. 

    You can then minimize the $\chi^{2}$ in the same way, only again with the matrix objects being aggregates of the per plane objects:
        \begin{align} \label{eq:deltafull}
            \Delta \vec{p}_{0} = \sigma_{p_{0}} \tau^{T}\sigma^{-1}(\vec{x}-\vec{p}),
        \end{align}
        %
        \begin{align} \label{eq:covfull}
            \sigma_{p_{0}} = [\tau^{T} \sigma^{-1} \tau ]^{-1},
        \end{align}
    where $\tau$ is the combined transport matrices from the individual 5x5 matrices, a 5Nx5 object.

    The unmeasured parameter errors of infinity still come into play in the final calculation in the same was as before. Because however these matrix objects are very large, and the tracking must have a certain amount of speed in order to keep up with data, it is useful to reduce the size of these matrices. (It also makes things easier programming wise. Note that there are other some other ways to speed things up, specifically the banded inversion method as described in reference \cite{trajfit}. This method was not used in favor of getting the code working in the simpler form in the first place, but it is a possiblity in the future to use this technique to speed things up even more.) It suffices to simply remove all rows and columns where said infinity values exist in the error matrix. This is mathematically equivalent to inverting the error matrix with the infinities included, which make all rows and columns where they exist go to zero. The associated unmeasured parameter rows in the residual vector and transport matrices must similarly be removed. This results in an Nx1 residual vector, NxN error matrix, 5xN combined transport matrix transpose, which multiply against the 5x5 covariance out front to still result in a 5x1 fix to the starting parameters, and a scalar $\chi^2$ value. (Note that these element removals should be done just before the final calculation, and not higher up in the algrebra, otherwise plane correlations are not properly calculated.)

    By calculating the last two equations one can fit the track, acquire a $\chi^{2}$ describing the degree of the fit, determine how the track parameters can be improved at the starting point, and calculate errors on those starting parameters. This algorithm can be iterated a number of times to get a best fit track until successive iterations produce no improvement, where usually 3 or 4 iterations is enough. Note that there is remarkable robustness with respect to the initial starting parameters in fitting the track. Of course if the initial starting paramaters are too poor, then the fit will not converge.
 % Math stuff


\section{In the gm2 Framework}

  Depending on the level to which a user wants to utilize or develop the tracking code, they may choose which branches of the gm2 framework to checkout. If just fitting tracks and viewing plots, then just gm2tracker on the develop branch may be checked out. If doing further development, it can be a good idea to checkout the artg4, gm2dataproducts, gm2geom, gm2ringsim, gm2tracker, and gm2utils packages of the gm2 repository, and be on the feature/trackDevelop branch in all instances. Every so often these branches are merged with the main develop branches, where the tracking code will work but might not have the latest code changes. 

  \subsection{Event Generation, Geometry, and Material}

    While not a direct part of the reconstruction and fitting code, there is relevant information with respect to event generation in the simulation upon which the reconstruction acts. Event generation before fitting is done using the mdc fcl files in gm2ringsim using the main simulation, where one can also include the tracker dummy plane geometry for truth comparison with the Geane fitting if so desired. These are built in the TrackerDummyPlane\_service and associated geometry files, and are included in the common mdc fcl files. If one wishes there are also a couple of modified mdc\#-geane fcl files available for use with reduced geometry. For stats reasons it is usually a good idea to include all 3 trackers when generating events.

    Fcl parameters exist for the StrawTrackerCadMesh\_service and Straws\_service in order to turn material on or off at will, ``materialTracker'' and ``strawMaterial'' respectively. The rest of the geometry has to be manually changed and rebuilt in order to remove material if one wishes. These include VacuumChamberCadMesh and the World, as well as the ``buildSupportPost'' option in strawtracker.fcl, ``buildTrolly'' in vac.fcl (where the associated material is hardcoded in), and ``trolleySupportMaterial,'' also in vac.fcl. One should make sure to perform reconstruction with the same material parameters as were used in the event generation for proper results. (These options are primarily for debugging the tracking. Near perfect results have been shown for tracking within a vacuum world, \href{http://gm2-docdb.fnal.gov:8080/cgi-bin/ShowDocument?docid=4876}{DocDB 4876} and \href{http://gm2-docdb.fnal.gov:8080/cgi-bin/ShowDocument?docid=4894}{DocDB 4894}.)


  \subsection{Reconstruction Flow}

    The overall tracking infrastructure and reconstruction flow can be seen in Figure \ref{fig:Infrastructure}. Data coming from simulation or the real experiment, are turned into art objects upon which the tracking framework acts. The RunGeane.fcl file (detailed more below) performs the entire chain within the blue box in Figure \ref{fig:Infrastructure}, excepting the track extrapolation stage. Within the reconstruction flow objects called Track Candidates are produced, which are the input to the Geane fitting code.


\begin{figure}[]
\caption{Shown is the infrastructure flow for the entire track reconstruction chain. In the green blocks are the sources of track data to fit, either from Geant simulation, or real data. In blue is the offline reconstruction block. Straw digits are formed in the digitalization step, those are then calibrated and grouped into time islands, clusters, and seeds which then combine to form track candidates. It is these track candidates which are the input into the Geane fitting code (and other future fitting code). The fitting code then outputs tracks which the track extrapolation stage will run over. This picture is taken from one of Tammy's talks. Note that there is some iteration here that is not shown.}
\centering
\includegraphics[width=0.9\textwidth]{TrackInfrastructure}
\label{fig:Infrastructure}
\end{figure}

\begin{figure}[]
\caption{Shown is the Geane fitting code flow. See the text for a thorough explanation of this flow, and the specifics of each box. GEANEArtRecords are created in the GeaneReco producer module, and are passed by pointer and reference to the different utils files which then updates them for the different fitting modes and with the fitting results.}
\centering
\hspace{15mm}
\includegraphics[width=0.9\textwidth]{NewGeaneFlow}
\label{fig:NewGeaneFlow}
\end{figure}


    The Geane fitting specific flow can be seen in Figure \ref{fig:NewGeaneFlow}. There is a number of files involved in the production of fitted tracks. Detailed information regarding all of these files is given below, but here is provided a shorter summary. Track candidates from upstream are used as the input to the producer module GeaneReco, which produces TrackArtRecords and TrackStateArtRecords, which are then passed to the extrapolation or refinement stage. GeaneReco also produces GEANEArtRecords, which is the data product that is used and updated throughout the fitting process until the track converges or fails, and may be passed through filters or to analyzers at the end of the fitting. There are a number of utils files that the Geane fitting utilizes. GeaneReco directly links to the GeaneFittingUtils which contains a number of methods for the different fitting modes, as well as the general fitting loop which links to the GeaneFitter class, and iterates until the track fitting succeeds or fails. GeaneFitter provides a $\chi^{2}$ and improvement to the track as described in the Formalism section. GeaneFittingUtils links to GeaneParamUtils and GeaneLRUtils. GeaneParamUtils deals with any code that modifies parameters regarding the fit, both setting up the initial fit parameters, and calculating the predicted parameters of the fit within Geant4 using the error\_propagation routines. GeaneLRUtils deals with any code regarding the left and right information of the track for fitting.
    
    Note that after a single iteration, one pass through GeaneFitter and one pass through the relevant fitting utils, a $\chi^{2}$ and an improvement for the track will be produced, but the track predicted parameters and objects will be based on the previous starting position and momentum guess before it has been updated, and so do not correspond to the fitted track. It's necessary to perform at least a second iteration for this reason, and is why no track converges under the strict criteria in under 3 iterations.

    Also worth mentioning is that the GEANEArtRecords are passed by reference or pointer throughout the various fitting files, allowing them to be updated along the way in a natural manner. At the same time, many methods within the fitting code return ints describing whether the track fitting has failed at different stages to various reasons, where a value of 0 signifies success at that stage. A returned int with a non-zero value might for example signify that too many digits were included in the initial track candidate, or that Geant is having tracking issues, etc.


  \subsection{fcl File Specifics - RunGeane.fcl} 

    Reconstruction is performed using RunGeane.fcl in gm2tracker on the events generated from somewhere higher in the chain. (There is also a RunGeane-midasdata.fcl file which is more tuned to running on real data, with certain fcl parameters set appropriately and some modules left out of the chain.) It is important to explain some fcl parameters necessary for the reconstruction. First, there is a fcl parameter ``useSD'' for the Straws geometry service to turn off the straw sensitive detectors so that in the reconstruction phase hits are not regenerated. If this is not included, it will default to true and cause crashes in the reconstruction. Secondly, the RunGeane.fcl file loads all 3 trackers necessary for symbol and name definitions, but such that tracker 0 (or 18) is rotated such that the tracking planes are parallel to the global geant X axis (using the rotateArcTracker fcl parameter for the Arc service). This is done to avoid the issue of error propagation instability close to the Z axis, as detailed in \href{http://gm2-docdb.fnal.gov:8080/cgi-bin/ShowDocument?docid=4567}{DocDB 4567}, while at the same time still observing the correct azimuthally symmetric 2D field for the tracks. Track hits are rotated from their separate tracker frames to this one reconstruction frame with the GeaneWorldTracker[0], GeaneWorldTracker[12], and GeaneWorldTracker[18] transforms defined at the bottom of StrawTrackerCadMesh\_service.cc, and as detailed in the Coordinate Systems section. (In the future, if fields or geometries are not identical between trackers, the details here will have to be improved after changes have been made.) As a reminder there are material fcl parameters for the straws and straw trackers which should be the same as were used in the event generation. The RunGeane.fcl file also includes some analyzer and filter modules by default, which are detailed further below, and can be turned on and off at will.

    While the pieces of the chain before the track fitting (digitalizing, seeding, clustering, t0 finder, and track finding) are distinct and separate from the material detailed in this document, it is important to understand at some level those stages and the different options which will affect the tracking results. See a collaboration talk by Tammy, \href{http://gm2-docdb.fnal.gov:8080/cgi-bin/ShowDocument?docid=5601}{DocDB 5601}, for a general overview on this. Here is a short summary, not all method or file calls are detailed. Most of these files and parameters should be left unchanged unless one is familiar with the subject matter.

    \begin{enumerate}

          \item{\bf{Digitalizer\_module.cc}} \\
          This module creates StrawDigits which will ultimately be combined to form the Track Candidates that the fitting will act over. It is the variables in this art record that are updated and correspond to real data, which straw was hit, the time of the hit, and the dca of the hit. The StrawDigits also contain a pointer to a struct of Monte-Carlo truth data, StrawMCDigit, which can also be accessed for track information. The Digitalizer module has several fcl parameters to change things like the gas drift time, drift model, allowing recorded negative times, and some particle cuts. By default it uses a Gaussian drift model which is called when the Digitalizer calls the DriftTimeModels.cc file, where it will gaussian smear the hit time from the MC digit time. These fcl parameters are located in digitalizerParams.fcl. The smearedDriftTimeSeed parameter should be set to 0 if one wants random results from run to run, which is necessary for grid jobs. The drift time model can also be set to the Garfield version, which will produce more realistic measurement parameters. If this is used the drift time calculator in the DriftDistanceCal module should also be set to the Garfield version. These more realistic hit data result in reconstructed tracks that haven't been fully analyzed yet, with a non-uniform p value distribution.

          \item{\bf{TimeIsland\_module.cc}} \\
          Groups the digits from the Digitalizer in time. Has some options for changing the time window width. fcl parameters are located in recoIslandParams.fcl. There is ongoing work in relation to this module for dealing with real data, where there is a much increased number of hits closer in time. The short summary here does not reflect the difficult and important task that this module has.

          \item{\bf{ClusterFormation\_module.cc}} \\
          Groups neighboring digits in the same view to form clusters. fcl parameters are located in recoClusterParams.fcl.

          \item{\bf{T0Finder\_module.cc}} \\
          Calculates a t0 for the track. Has an option for which model to use for calculating said t0. The default is using truthData if one is running the tracking over simulated hits. If one is looking at real data there are other options. It's important to note that the t0 for a track can be difficult to calculate, and the current goal is to combine information from the calorimeter to get this number. fcl parameters are located in recoT0FindingParams.fcl.

          \item{\bf{DriftDistanceCal\_module.cc}} \\
          Calculates the drift times and distances within the straws based on the calculated t0 from the previous module. Has options for different drift calculators and parameters, including a ``useTrueDigitT0'' parameter for using mc truth information. The drift time is calculated from the digit hit time (calculated in the Digitalizer) minus the digit t0. If true t0 is not used, then the dca smearing will not be perfectly gaussian. fcl parameters are located in recoClusterParams.fcl.

          \item{\bf{SeedFormation\_module.cc}} \\
          Groups neighboring clusters in the same module to form seeds. There are two fcl parameters for this module. The first is ``reconstructPosition'' which will fill seed nodes which form the seed, and essentially correspond to indidual digits - but which contain some left-right (LR) ambiguity information important for track fitting. The second is ``useGeometryLR'' which uses the straw geometry information to make an initial guess at the left-right choices for doublets within the track. fcl parameters are located in recoSeedParams.fcl.

          \item{\bf{Aside}}
          It's important to note that there is another iteration of the clustering module which is improved now that better drift times and distances have been calculated from the previous modules, using the same fcl paramters. This will almost certainly change in the future as the tracking code becomes more developed.

          \item{\bf{TrackFinding\_module.cc}} \\
          Groups time islands and digits based on their positions in space. There are two finders available for use, fcl parameter trackFinderName, the Simple Track Finder (SimpleTrackFindingUtils.cc) and the Long Track Finder (LongTrackFindingUtils.cc). The former just runs off the generated time islands. If you use this, then the second clustering module and seed formation module can be left out of the reconstruction chain. The latter, Long Track Finder, forms track candidates based on combining seeds and is the default used. It also has stricter fcl conditions, such as a default cut of 6 or more planes hit for the track. Both finders make use of the SimpleCircleFitter.cc class which fits a circle to the U and V digit positions to calculate a starting position and momentum guess for the track, which is currently necessary for the Geane fitting. The Long Track Finder also has a least squares minimizer which is unimportant for the Geane fitting. fcl parameters are located in recoFindingParams.fcl, which then also links to some other fcl files depending on the finder used.


    \end{enumerate}

  \subsection{geaneFitParams.fcl}

    Because the Geane track fitting is the subject of this document, it will go more into detail of the Geane fitter fcl parameters, located in the geaneFitParams.fcl file. Some of these parameters are very likely to change or disappear in the near future.

    \begin{enumerate}

      \item{\bf{module\_type}} \\
      The ordinary art producer parameter which tells art which cc file to run, this is always set to GeaneReco unless the filename changes.

      \item{\bf{G4EVERBOSE}} \\
      This parameter corresponds to the iverbose level present in the Geant4 error\_propagation files and can be manually set here, from 0 to 5. It is for the debugging of deeper level Geane fitting code. Note that unless Geant is compiled with the flag by the same name setting this verbose level will do nothing. Since currently we utilize our own slightly modified error\_propagation files copied into the gm2tracker/gm2Geane directory, it would be possible to remove the dependencies on this Geant4 build flag. Since however this will slow down the tracking it is the current choice to leave them in. 

      \item{\bf{trackingVerbose}} \\
      Geant4 tracking verbose level, from 0 to 5. This is always available but should only be used for debugging.

      \item{\bf{matrixDebug}} \\
      This boolean paramater turns on and off the copious matrix debugging output, used for deep level debugging. Note that these outputs are included in the log file and the log file threshold needs to be set to ``DEBUG''. In the future it might be beneficial for speed reasons to remove completely these debugging outputs. This parameter should always be turned off if looking at a large number of events.

      \item{\bf{numPassesWireFit, numPassesSeqFit}} \\
      These are parameters used for the Left-Right sequence checking routines for a full fit. numPassesWireFit is the number of passes that the track should be fit to the wire centers, before which the best sequences were passed to the main fit code which would iterate a number of times equal to numPassesSeqFit. It was observed in the past that 3 passes for each of these was sufficient for good tracking results. 

      \item{\bf{convergenceCriteria}} \\
      This is a scalar convergence criteria telling whether the track fitting has been successful or not, with a default value of .1. As long as the current iteration $\chi^{2}$ minus the previous iteration $\chi^{2}$ is less than this value, the track is considered to have converged, otherwise it continues iterating until it succeeds, fails, or is manually cut off at 10 iterations.

      \item{\bf{useCircleGuess}} \\
      This boolean parameter tells the code whether to use the circle fitter results for the track starting guess or not, called in the track finding part of the chain. It's default should remain true unless one wants to return to the uniform starting parameter smearing for debugging purposes.

      \item{\bf{lockLowDCAs}} \\
      This is a temporary fcl parameter which will take measured hits with small dcas and set the measured positions to the wire centers, at the same time locking the LR choices to the center. This will be necessary when fitting data due to the large measurement uncertainty of hits close to the wire, and at the same time speeds up the code with less LR choices to iterate over. It's default value of 0 corresponds to turning this functionality off, while a positive number corresponds to the radius in mm that dcas less than will be locked.

      \item{\bf{useNodes}} \\
      This boolean parameter sets whether to use the left-right information from the seed nodes, if activated. The recoSeedParams fcl parameter ``reconstructPosition'' needs to be set to true for this to work, and the parameter ``useGeometryLR'' should also be set to true, being that it is the only upstream left-right code currently implemented.

      \item{\bf{fitMode}} \\
      This is the most important fcl parameter for the Geane fitting. It defines what type of fitting mode you'd like to use for fitting the incoming track candidates. There are currently 4 separate fitting modes. ``truthLRFit'' only works with simulated data and will fit to the unambiguous LR UV values from the digits. ``wireFit'' will fit a track to the wire centers of the incoming digits, with a corresponding uniform error proportional to the gas diameter. ``mainFit'' will first do a wire fit to the track in order to make an initial guess at the LR choices for each hit, and will then fit to those LR choices. (The default main fit actually does a second wire fit in the middle where it first locks LR choices for doublets where the LR choice is guaranteed to be known, which improves the tracking results slightly.) Finally there is the ``fullSeqFit'' which will first do the wire fit, and will then go about checking all LR sequences before doing full fits on the set of best sequences. (There is also some code to lock the left-right sides based on the doublets.) This LR procedure is detailed lower in this document.

      \item{\bf{useTangentLR}} \\
      This fcl parameter links to code which does a slightly more advanced locking of the doublet left-right geometry include track tangent information, defaulted to false, and written by Joe Price.

      \item{\bf{rseed, yPosChange, zPosChange, xMomChange, yMomChange, zMomChange}} \\
      These are the uniform smearing bounds for the starting position and momentum of the track from truth, units of mm and MeV. These are largely unused now since the circle fitter provides a level of smearing automatically when it's used, but should be kept for future debugging purposes. The rseed parameter is the number seed for the random number generator. The default should be 0 for the root TRandoms unless one wants to reproduce results.

    \end{enumerate}


  \subsection{Reconstruction Files}

    Here will be given a summary of the different core files used in the Geane track fitting as shown in Figure \ref{fig:NewGeaneFlow} and explained up above, as well as extensive detail where necessary. 


    \begin{enumerate}

      \item{\bf{GeaneReco\_module.cc}} \\
      This file is the producer module for the track fitting reconstruction. It pulls in upstream data products, TrackCandidateArtRecords, and produces TrackArtRecords, TrackStateArtRecords, and GEANEArtRecords. The GEANEArtRecords are what's passed through the various fitting utils files until the track fitting succeeds or fails, and from which the TrackArtRecords and TrackStateArtRecords are derived from. GeaneReco directly links to GeaneFittingUtils where the track fitting is actually done.

      \underline{Methods}

        \begin{itemize}

          \item{produce} \\
          General overriden produce method. Calls main track fitting code on an event by event basis. Produces data products for downstream use.

          \item{trackFitting} \\ 
          Short method which first calls down through GeaneFittingUtils into GeaneParamUtils to set up the initial parameters for fitting, and then calls the method corresponding to the particular fit mode specified in the fcl file. Also has a method call for checking some failure modes after fitting.

        \end{itemize}

/////////////////////////////////////////////////////////////////////////////////////
      \item{\bf{GeaneFittingUtils.cc}} \\

      \underline{Methods}

        \begin{itemize}

          \item{setXPositions} \\
          \item{truthLRFit} \\ 
          \item{wireFit} \\ 
          \item{mainFit} \\
          \item{fullSeqFit} \\
          \item{fittingLoop} \\ 
          \item{checkExtraneousFailureModes} \\ 

        \end{itemize}
/////////////////////////////////////////////////////////////////////////////////////

/////////////////////////////////////////////////////////////////////////////////////
      \item{\bf{GeaneFitter.cc}} \\
      This file consists of the main track fitting chi2 algorithm. It multiplies measured parameters, predicted parameters, error matrices, and transport matrices together to produce a chi2 for the track and an improvement to the starting paramters. It also reads in the measured hits errors in order to properly fit the track. See the Formalism section for a detailed description of what this class really does. There is a fcl parameter matrixDebug which can be used to turn on or off the many large matrix cout debugging statements.

        \begin{itemize}

          \item{TrackCorrelation} \\
          The main matrix multiplication routines.

          \item{makeHybridErrMat} \\
          A method in order to create a hybrid wire / U or V error matrix for L/R sequence checking. Only called once per event.

          \item{sequenceChecking} \\ 
          The main L/R sequence checking method for an individual event. This method is called up to thousands of times as each U or V sequence is checked. See the Left-Right section further below for more detail on what this method does.

          \item{convertToGeVcm, convertToMeVmm} \\
          Methods for converting Eigen 5 parameter vectors from GeV cm to MeV mm and vice versa.

        \end{itemize}
/////////////////////////////////////////////////////////////////////////////////////

/////////////////////////////////////////////////////////////////////////////////////
      \item{\bf{GeaneParamUtils.cc}} \\
      only file that links to geant
GeaneReco then tracks particles along their average trajectories in the Geant4 world, producing the matrix objects necessary for fitting as described in the Formalism section. 

      \underline{Methods}

        \begin{itemize}

          \item{setXPositions, setCoordMap} \\
          \item{setupParams} \\ 
          \item{errorProp} \\ 
          \item{calcMeasuredParams} \\ 


        \end{itemize}
/////////////////////////////////////////////////////////////////////////////////////

/////////////////////////////////////////////////////////////////////////////////////
      \item{\bf{GeaneDummyUtils.cc}} \\

      \underline{Methods}

        \begin{itemize}

          \item{fillDummyHits} \\

          \item{getTrackDummyHits} \\ 
          \item{fillCS, fillStationStr} \\ 
          \item{fillDummyHitInfo} \\ 
          \item{fillDigitHitInfo} \\ 
          \item{fillTruthParams} \\ 
          \item{createStartGuess} \\ 
          \item{fillLRFromTruth} \\ 
          \item{checkLRAgainstTruth} \\ 


        \end{itemize}
/////////////////////////////////////////////////////////////////////////////////////

/////////////////////////////////////////////////////////////////////////////////////
      \item{\bf{GeaneLRUtils.cc}} \\

      \underline{Methods}

        \begin{itemize}

          \item{lockSmallDCAErrors} \\

          \item{lockSmallDCACenters} \\ 
          \item{fillLRFromFit} \\ 
          \item{fillLRFromGeom} \\ 
          \item{setUnknownSides} \\ 
          \item{fillLRFromNodes} \\ 
          \item{getUVUnkowns} \\ 
          \item{getTopSequences} \\ 
          \item{modifyMeasuredParams} \\ 
          \item{getSequenceSides} \\ 

        \end{itemize}
/////////////////////////////////////////////////////////////////////////////////////

      \item{\bf{GeaneTrackUtils.cc}} \\
      This utils file contains a number of useful methods for getting track information on wire planes, checking whether a plane is a U or V plane, and a couple of Eigen variables used in converting from an XYZ frame (with planes staggered parallel in the X direction) to the Geane XUV frame.

      \item{\bf{GeaneEigenStorageUtils.cc}} \\
      Art data products cannot permanently store eigen objects. This utils file converts Eigen objects to c++ vectors when storing said objects into the art record, and vice versa when reading from the art record. Written by James Mott.


move/change these methods to the correct area

          \item{InitializeGEANE} \\
          Geant4 GEANE (error\_propagation) initialization method called by the constructor, once per run.

          \item{modifyMeasuredParams} \\ 
          Short method for LR sequence checking, modifies measured parameters based on particular sequence.

          \item{fittingLoop} \\ 
          Method which contains the fitting loop code which successively calls errorPropagation in the same file and the GeaneFitter for fitting the track. The angleCorrection method is called each iteration to update the UV measured positions and errors depending on the track momentum. (Before it is called the errors are reset so that they are not double corrected.) There is also a method parameter for updating the LR choices based on the fit at each iteration, which is only used when doing a ``mainFit'', which improves the final tracking results. This method will loop inside itself until convergence is achieved or the max number of iterations is reached (or fitting fails).

          \item{checkExtraneousFailureModes} \\
          This is a short method used for checking if track results might be off because of other extraneous reasons.

          \item{errorPropagation} \\ 
          This method tracks particles through the detector using Geant4 error propagation routines with the correct geometry and field. It builds transport matrices, error matrices, and predicted parameters which are the objects used for fitting the track. It tracks on a plane by plane and step by step basis. These routines can be used to track particles forwards and backwards, where the forwards tracking is used in the code. Changing to backwards tracking would be very non-trivial. Add more detail here later..

          \item{angleCorrection} \\
          Method to iteratively correct measured parameters from a radial DCA value to a U or V value based on the momentum of the track and approximating a constant field within the straw. Also corrects the errors using a simple straight line approximation which is good enough. See appendix section for more details on how this works.






  \subsection{Analyzers and Filters}


      \item{\bf{GeanePlots\_module.cc}} \\
      The main analyzer/plotting module which runs on the output GEANEArtRecords produced upstream. This creates many plots including chi2 distributions, p value distributions, number of iterations, track parameter, residuals, pulls, etc. Truth information for these plots is currently necessary as written, but can be take out in the future for real data. There are some fcl paramters to make cuts on different parameters. There are also some other plotter modules which are similar but much reduced in scope. reference geanePlotsParams.fcl file here.

      \item{\bf{GEANESingleEventViewer\_module.cc}} \\
      include a plot or two of what the single event viewer can show, how to use it (specify events or not)

      \item{\bf{GEANEProcessSelection\_module.cc}} \\
      mention how this insn't fully devloped yet

      \item{\bf{GeaneFailedPlots\_module.cc}} \\
      explain failed plots mode, included plot

      \item{\bf{GeanePoorEvent\_module.cc}} \\
      selection on bad events and why

      \item{\bf{geane plots macro}} \\
      I think also include some notes about this guy

    \end{enumerate}


  \subsection{Other useful fcl files}

    There is a number of other fcl files which are currently included in gm2tracker which are useful for various analysis or fitting purposes. These are relatively simple and straight forward fcl files which can be deleted, combined, or changed however the user wishes depending on what they are looking at. They are typically used on produced sets of GEANEArtRecords.

    \begin{enumerate}

      \item{\bf{RefitGeane.fcl}} \\
      This file simply refits a previously produced set of track candidates, typically used for refitting failed or bad events from a previous attempt at fitting. One can make changes to the fitting code, refit, and compare to previous results.

      \item{\bf{makeGEANEPlots.fcl}} \\
      This file will call the main GeanePlots analyzer module for producing plots about the GEANEArtRecords, useful for when one wants to add or remove plots after tracks have already been produced.

      \item{\bf{ProcessSelectionGeane.fcl}} \\
      This file will call the filter on different physical processes that events have undergone within the tracker region, if one wishes to separate events based on this criteria.

      \item{\bf{makeGeaneSingleEvent.fcl}} \\
      This file will take produced GEANEArtRecords and call the single event analyzer on the specified set of events. This is useful for analyzing single events to see where rare errors crop up, if tracks are kinking, for debugging purposes, etc. This should not be used for a large number of events.

      \item{\bf{filterTracks.fcl}} \\
      This file will filter on whether art events successfully produced a GEANEArtRecord or not (succeeded or failed), which is useful for reducing the size of art files which contain many events that do not hit the tracker or don't hit enough planes.

      \item{\bf{fileMerge.fcl}} \\
      This very general file will simply combine the set of art files that are passed to it, useful for reducing the number of files one has to deal with.

    \end{enumerate}





\subsection{Code details}

Building and multipilying transport and SC2SD matrices
Converting things to eigen objects

JacobianToUV stuff - or combine with other coordinate transform stuff
Matrix accumulation/reduction
Notes on loop indices, start points and such



\subsection{Code peculiarities / notes}

Where to structure this in the file?

Notes on coordinate systems I use - UV pointing outward vs XYZ, XUV x being forward, UVW in geante src, maybe IJK/TUV in one paper-prob not, GeaneTrackerWorld[0,12,18], rotateArcTracker (bit more specific), magnetic field access coordinates?, JacobianToUV

accessing the field in a specific way



\section{Left-Right}
\label{sec:LR}

	This section will describe the fitting modes that can handle determining the left-right sides of the hits within the fitting. The first and shorter section describes the ``mainFit'' fit mode and the second and longer section describes the ``fullSeqFit'' fit mode. Some of this information is repeated from the GeaneFittingUtils.cc section describing these fit modes. It is important to remember that while some left-right information for the track may be determined upstream, most of it cannot until a fit of some sort is completed. For us that is the wire fit, which does not require any left-right information in order to work. The wire fit serves as the starting point for determing the left-right sides of the hits since the overall fit will approximately get the left-right sides correct, though not always. Even a single hit side being incorrect can throw the fit off, which is usually readily apparent with resulting high $\chi^{2}$s or low p-values. Sometimes this incorrect side (or sides) can be determined by looking at the individual plane $\chi^{2}$s and finding the outliers, though this is not always the case. Due to the smearing of the real dcas from the straw resolution, sometimes it can be very difficult to determine the left-right sides in all hits. There is planned to be a refinement stage in the future, which will be tasked with adjusting the track that is to be fitted, both dealing with left-right sides and removing or adding hits to the fit. Until such a time however, we are unable to fit all tracks correctly with 100\% fidelity due to this left-right ambiguity problem.

	\subsection{mainFit}

	  First a wire fit is completed with a couple of iterations. Then a fit is performed where each iteration takes as it's hit sides wherever the previous fit ended up. Something like 2/3rds of the tracks converge nicely with this method, but not all. (In the current iteration of the code there is also an intermediate second wire fit, where the known sides from the doublet geometry and track angle are set, which helps salvage a couple of tracks.) If the left-right sides are only taken from the wire fit a single time, instead of being updated each iteration, then less tracks converge.

	\subsection{fullSeqFit}

      This fitting mode takes the longest time to fit a track, because it checks every single LR combination for all hits in order to find the best track. It starts by first doing a wire fit for all hit digits in order to get approximate track objects (transport matrices, error matrices, predicted parameters) used for making approximate guesses at how good different LR combinations are. Second, one can lock certain hits to left, right, or center depending on how the user is fitting (locking LR from known geometry, small dcas, etc), and the track fitting will loop over the remaining unknown sides, checking each combination. In order to increase the speed of this code greatly, instead of checking all $2^{N}$ combinations directly, the separate $2^{Nu}$ and $2^{Nv}$ combinations are checked separately (leaving the other parameter sides as being set to the wires). See Figure \ref{fig:SeqCheckSteps} for a diagramatic view of exactly how this approximate fitter works for the sequence checking, and read the \hyperref[sec:GeaneLRUtils]{GeaneLRUtils.cc} section for detail on the methods that are used within the fullSeqFit.

			\begin{figure}[]
				\caption{This figure shows the steps used to determine an approximate set of the best U and V left-right sequences. Note that these steps are followed separately but identically for U and V. The equations here are identical to some of those given in the \hyperref[sec:Formalism]{Formalism} section. Step 1 is performing a wire fit, done once for all $2^{N}$ U or V combinations. The resulting generated transport, error, and covariance matrices along with the predicted parameters from the wire fit are saved for the following steps. Step 2 involves using those aforementioned matrix objects, but now with the measured positions in the calculation replaced with whatever particular left-right sequence we're checking. An approximate fix to the track is then calculated, which can be used in step 3 to determine a new set of approximate predicted parameters. These approximate predicted parameters are then combined with the measured left-right sequence we're checking in step 4, along with a hybrid error matrix (where if we're checking U sequences, then the V planes have wire errors, and vice versa) to calculate an approximate $\chi^{2}$ for that particular sequence. All U and V sequences are checked and the best 10, with the lowest $\chi^{2}$s, of each are saved. The arrows show where matrix objects are used from one step to another. This method while still slow for many combinations, especially high N tracks, is pretty fast because it's simple matrix multiplication over and over again, without actually Geane fitting eaching $2^{N}$ combination. (And no inversions.) Note that while not as effective, it is possible to jump from step 1 to step 4, where the predicted parameters remain as those predicted from the wire fit, and calculate an approximate $\chi^{2}$ for the left-right sequence. This has the benefit of being faster by skipping steps 2 and 3 but fails for more events, and so isn't as useful.}
				\centering
				\includegraphics[width=0.8\textwidth]{SeqCheckSteps}
				\label{fig:SeqCheckSteps}
			\end{figure}

 	  Once the set of best U and V sequences have been determined with the approximate fitter, they are combined together and fitted with the full Geane error propagation fitting. For a small sample of fitted tracks, Figure \ref{fig:PositionOfBest} shows where in the set of best 10 saved $\chi^{2}$s for the different sequences the actual truth sequence is. (For a larger sample of tracks, this plot would fill out more with less empty bins.) It can be seen that the approximate fitter actually guesses the correct U and V sequences a large percentage of the time. When combining the U and V sequences together and full fitting, it is beneficial to scan over some smaller set instead of the full 10 by 10 grid. The red line in the figure separates out a set of 15 tracks to full fit, which results in high fidelity of fitted tracks. If one wishes an even smaller set of tracks can be full fitted, where more tracks will result in poor fits, but the speed of the tracking improves significantly. See \href{https://gm2-docdb.fnal.gov/cgi-bin/private/ShowDocument?docid=6800}{DocDB 6800} for a quick study of track fitting speed vs the level of results, where one should pay attention to the number of tracks within the 0 bin, representing poorly fit tracks coming from incorrectly chosen left-right sequences. 

 	  It is also possible to cut short the sequences that are being full fitted. When there is a couple of very similar sequences, possibly with small dcas, the track fitting results in similar $\chi^{2}$s. Usually soon after there is a large jump in the full fitted track $\chi^{2}$ as some left-right choices have been changed, revealing that that particular sequence and the ones after are incorrect. For example the $\chi^{2}$s might go as 10.2, 11.4, 10.3, 38.9..., and it is at this large jump that one can make a reasonable assumption that the best sequence has been found. This of course does not work perfectly all of the time but it is a good way of eking some more speed out of the tracking. The user can configure the fitting (from base fcl parameters to options such as describe here) to optimize vs fit results or speed however desired. This should be combined and studied with the future refinement stage. One might wish to do faster fits to start and refine the poorly fitted tracks before doing more thorough sequence checking and the like.

			\begin{figure}[]
				\caption{The fullSeqFit mode saves the smallest 10 $\chi^{2}$s from the approximate fitter for the ``best'' U and V sequences. For the set of 10 best U sequences, the X bins in this plot represent the location of the true U sequences, and similarly for V. For example, if the 0 bin is filled then the U sequence with the smallest $\chi^{2}$ is the true sequence, if the 1 bin is filled then the U sequence with the second smallest $\chi^{2}$ is the true sequence, etc. It can be seen that the approximate fitter does a very good job of actually guessing the true U and V sequences with the bottom 3 bins holding most of the events. Some sequences do end up in the higher bins where the approximate fitter has not done so good of a job at determing the true left-right sequences, including some events which end up in the overflow bins where the approximate fitter has simply failed. The angled red line describes a region of combined sequences to perform the full fitting over in order to try and find the combined true left-right sequence. Note that even then the true sequence may not have the smallest $\chi^{2}$ in the final fitting due to the smearing of dcas. Note also that hit digits with very small dcas will have very similar $\chi^{2}$s and fill neighboring bins, though this fact is ignored when locking small dcas to the wire centers.}
				\centering
				\includegraphics[width=1.0\textwidth]{PositionOfBest}
				\label{fig:PositionOfBest}
			\end{figure}




\section{gm2Geane package in artg4 - eloss, brems, ionizations, msc}

See many docdbs by Nick Kinnaird on geane updates for more info if wanted. Summary of stuff here.


  There is a folder under artg4/gm2Geane, where slightly (but importantly) modified geant/geane code that the geane track fitting uses is located. Most importantly is that the reconstruction was taking too much energy away from the particle during propagation due to bremsstrahlung, so that has been removed. (Thanks James for help with this.) (Probably explain flow of file - extrapolator tables etc.).

  Besides those associated files, all files from the geant folder ``error\_propagation'' have been copied into this directory as well, with the only non-name changes lying in gm2GeaneFreeTrajState.cc. (Many files had to be copied due to naming/linking - some could probably have been omitted but I decided to copy the whole folder.) Ionization and multiple scattering errors have been modified according to Lavezzi thesis (explain more) - with no improvement in results unfortunately. (Probably turn defaults back to original code (still gm2Geane) with the option to turn things on somehow if desired - at least leave them in for future tuning.)

Should I go more in depth about the error\_propagation/gm2Geane source code? Probably - at least for certain main files.


Add many plots and pictures of things - pulls, single events, tracks, etc etc.

\begin{figure}[]
\caption{Shown here is the energy loss between the first and last hit in the tracker from simulation. (Momentum magnitude difference.) Sources of energy loss come from ionization and bremsstrahlung processes, which account for the long Landau tail running off to infinity. The distribution has a mean of approximately 220 keV and and peak centered at about 150 keV, which is reasonable for the material composition of our trackers and tracker gas. More than 50\% of the energy loss comes from the mylar walls of the straws as seen in the simulation.}
\centering
\includegraphics[width=1.0\textwidth]{eLoss}
\label{fig:eLoss}
\end{figure}

\begin{figure}[]
\caption{It was noticed during the course of debugging that the Geant4 error\_propagation routines were consistently removing too much energy on average from all tracks as they passed through the tracker during reconstruction. It was discovered that the Geant4 tables were taking out too much energy due to bremsstrahlung processes and it was decided to remove this effect. The left plot shows in a red the true particle momentum as a function of X distance through the tracker, and in black the reconstructed momentum before bremsstrahlung was taken out of the reconstruction for a single event. The differing slopes signify the problem. The right plot shows the same event but with bremsstrahlung taken out. Notice scale change, and that the reconstructed momentum aligns much more readily with the truth.}
\centering
\includegraphics[width=1.0\textwidth]{bremComparison}
\label{fig:bremComparison}
\end{figure}


% /////////////////////////////////////////////////////////////////////////////////////
\section{Coordinate Systems}

  The only material not covered here in full detail is the mathematical explanation of reference frames, for which one should see the reference papers. It suffices to summarize as follows: GEANE objects (matrices and parameter vectors) are defined and calculated in the Geant4 source code in the free particle system. Then there is an intermediate surface system defined in XYZ, as the Geant4 surface trajectory system must be defined in an orthogonal coordinate system, before which parameter objects are converted to the most natural detector system of XUV. (Important note - be very careful with coordinate system variables, letters are reused between different papers and code bases with different configurations and meanings constantly.)


\begin{figure}[]
\caption{Shown here is a picture of the 3 trackers in the world geometry in their approximate positions. First note the world coordinate system shown in the bottom left of the plot, where the origin lives at the center of the ring. Tracks are then generated and read out from the trackers in their three world positions in red. Due to the reconstruction bug where tracks improperly reconstruct if their momenta is too aligned with the globabl Z axis, the reconstruction rotates the entire arc including tracker 0 to the blue position, where planes are parallel in X, and this problem is avoided. Track parameters from the 3 positions are then rotated to this reference frame by the amounts shown on the plots for the reconstruction stage, and at the end are rotated back.}
\centering
\includegraphics[width=1.0\textwidth]{WorldCoordSys}
\label{fig:WorldCoordSys}
\end{figure}

\begin{figure}[]
\caption{This picture shows the coordinate system in which the the Geane track reconstruction is performed, in relation to the world coordinate system. The origin remains at the center of the ring, with the tracking planes parallel in X in the reconstruction, going forward in number. Y is vertically up, and Z is horizontally to the right. U and V are defined such that they have greater values with higher radii and increasing straw number.}
\centering
\includegraphics[width=0.6\textwidth]{GeaneCoordSys}
\label{fig:GeaneCoordSys}
\end{figure}


$1/p, \lambda, \phi, y_{\perp}, z_{\perp}$, free system

$1/p, py/px, pz/px, y, z$, surface system

$
\begin{pmatrix}
u \\
v \\
\end{pmatrix} =
\begin{pmatrix}
-\sin{\theta} & -\cos{\theta} \\
\sin{\theta} & -\cos{\theta} \\
\end{pmatrix}
\begin{pmatrix}
y \\
z \\
\end{pmatrix}
$, yz to uv matrix, where $\theta$ is $7.5\degree$, the 5x5 transformation is just a 1 in the top left corner, and then this matrix in the remaining 2 diagonal blocks

$1/p, pu/px, pv/px, u, v$, uv system

% /////////////////////////////////////////////////////////////////////////////////////




% it seems I have to cite the references in order for them to be added to the paper - some weird build issue, just do it here until I cite them naturally in the paper
\cite{jacob}, \cite{energyloss}

\printbibliography


\section{notes and other}

 Due to the CBO of the beam, 1 tracker accounts for ~90\% of the physics, and a second tracker accounts for much of the remaining. (Find/learn sources for this to explain a bit better..)

  starting error matrix

  linking equations to code variables

  initial errors into tracing/error propagation?

  mention that it's possible to make a Kalman filter which uses the geane matrix objects? and that I didn't do so because we wanted two separate methods for the tracking originally and someone else was working on the kalman filter? (and the chi2 minimization was used in the last experiment)

  make a subsection on the left-right ambiguity work 

  negative chi2 stuff

  note on older branches for development at all?

  \subsection{Kalman Filter}

    There is some precedent for using the error propagation matrix objects within a Kalman filter as detailed in the Lavezzi thesis \cite{Lavezzi}, but due to the circumstances following the creation of this code that technique was not followed. At the beginning of development, a Kalman filter was being started in tandem by others, which has since fallen by the wayside. In order to have two separate methods for fitting however, the least squares global minimization was chosen. The simpler method was also chosen as the author was unfamiliar with track fitting and coding in general when beginning development.



\section{Acknowledgements}

Should I put an acknowledments section in? (Rob, Rob, James, Joe)
From a technical document perspective maybe not?


\section{Appendix}

\subsection{Angular correction}

  Because our straws don't in fact measure U and V coordinates directly, but instead distance of closest approach radii deriving from measured hit times, in order to utilize the minimization procedure on measured track parameters these radii must first be converted to U and V parameters. (And similarly for the U and V errors.) (Note that the future Kalman filter will not be subject to this disadvantage.) This is done in the angularCorrection method at the bottom of the GeaneReco module. The wires that are hit as well as the initial raw measured errors are stored indefinitely, and it is upon these objects that corrections are made. These corrections will be dependent on the angle of the track, so it's imporant to note that during each successive iteration, the ``measured'' parameters are adjusted by the latest ``predicted'' momenta. It was found that for the error correction, a simple straight line correction was sufficient for ideal results. For the position correction, it was found that a constant field correction for curved tracks was sufficient.

  To calculate these corrections, first the momentum perpendicular to the straw measurement axis can be ignored. Because the positron tracks curve in only one direction through the tracker, one needs to calculate the correction depending on whether the track went to the left or right side of the wire. See Figure \ref{fig:angularCorrection} for a pictorial representation of the problem. The calculation of the right side correction follows, with the left side correction being calculated in a similar manner.

\begin{figure}[]
\caption{Shown here is a positron passing through a straw. The desire is to convert the measured parameter d into a U or V position, which can be done by approximating the particle trajectory as a circle in a constant magnetic field over the course of the straw and using trigonometry. Sizes and angles are exaggerated.}
\centering
\includegraphics[width=1.0\textwidth]{angularCorrection}
\label{fig:angularCorrection}
\end{figure}

An equation with the single unknown parameter being of interest can by approximated by
$$(r+d)^{2} = r^{2}+u^{2}-2ru\cos(90+\theta),$$ where the 90 degrees is approximate for large curvature of tracks , and we want to solve for u. The angle $\theta$ can be determined from 
$$\hat{z_{\parallel}} \cdot \hat{p_{\parallel}} = \cos{\theta}, \theta = \cos^{-1}{\frac{p_{\parallel}}{p}},$$ where p parallel is the momentum parallel to the z parallel axis at the wire plane and can be determined within the code. Using a trig identity and solving for u gives
$$u = -r\sqrt{1-(\frac{p_{\parallel}}{p})^{2}} + \sqrt{d^{2} + 2dr + r^{2}(1-(\frac{p_{\parallel}}{p})^{2})},$$ for the right side correction, and 
$$u = +r\sqrt{1-(\frac{p_{\parallel}}{p})^{2}} - \sqrt{d^{2} + 2dr + r^{2}(1-(\frac{p_{\parallel}}{p})^{2})},$$ for the left side correction. (Corrections to v are identical.) The radius of the particle circle can be calculated from the circular momentum and magnetic field at the predicted hit position. The straightline correction is done simply using the Pythagorean theorem in a simpler manner, with the correction to the errors then being
$$\sigma_{uv}' = \frac{\sigma_{uv}}{\sqrt{1-(\frac{p_{\parallel}}{p})^{2}}}$$.



\begin{longtable}{|p{16cm}|}
% \scriptsize
\caption{GEANEArtRecord.hh active variables. This is subject to change. GEANEArtRecord contains vectors of variables on planes, as well as larger objects containing information about the whole track. Note that Eigen matrix objects cannot be stored into art data products. For this reason, and for minimal code changes, it was decided to add a data object for each Eigen member, made up of vectors with the word ``Data'' tagged at the end. The data objects are saved when creating GEANEArtRecords using the GeaneEigenStorageUtils file. In analyzers accessing the GEANEArtRecords, these data objects are then swapped into the Eigen objects using the same utils file.}
 
\label{tab:artRecord}

% \begin{tabular}{|p{16cm}|}

  \\ \hline

enum GeaneHitSide \\ 
\textit{An enum for which side of the wire the track hit is guessed or calculated to have passed. Options are gLeft, gRight, gCenter, gNA\_side, and gUnknown. The first three are self explanatory, gNA\_side simply means there wasn't a hit, and gUnknown is used within the LR sequence checking part of the fitting code where any hits with said value are looped over to try and find the best choice.}

  \\ \hline

art::Ptr \textless{} gm2strawtracker::TrackCandidateArtRecord \textgreater{} candidate \\
\textit{One track corresponding to one candidate corresponding to one GEANEArtRecord for the whole track.} \\ \hline

std::vector\textless{} art::Ptr\textless{} gm2truth::GhostDetectorArtRecord \textgreater{} \textgreater{} dummyPlaneHits \\
\textit{Associated dummy plane hits on planes aligned with straw wires, vector consists of hit dummy planes corresponding to hit wire planes (if a straw plane was skipped but that dummy plane was hit, it is not included in this vector. Vector has size N = num hits in straws that form the track +1 for the 0 plane.)} \\ \hline

int failureMode \\ 
\textit{Different failure modes for failed track reconstruction, 0 means it passed.} \\ \hline

double chi2 \\ 
\textit{Chi2 for whole track.} \\ \hline

std::vector\textless{}double\textgreater{} chi2Planes \\
\textit{Individual chi2s on each plane, vector consists of hit planes with size N. The sum of the chi2s on the planes do not add up to the total chi2, since they don't include the correlations, but get close and can still be used as a measure of goodness of fit at that plane.} \\ \hline

int numIterations \\
\textit{Number of iterations to converge.} \\ \hline

unsigned int dof \\
\textit{DoF of track = number of hit planes - 5 track parameters.} \\ \hline

double chi2DoF \\
\textit{chi2/dof} \\ \hline

double pValue \\
\textit{Fit pValue for whole track.} \\ \hline

std::vector\textless{}double\textgreater{} startingTrackParameters \\
\textit{Starting parameters for track: size 6, 3 position then 3 momentum, x y z px py pz, best starting parameters updated after each iteration, starting parameters x position defined before first hit.} \\ \hline

int trackNumPlanesHit \\ 
\textit{Total number of planes hit.} \\ \hline

int trackFirstPlaneHit \\ \hline

int trackLastPlaneHit \\ \hline

std::vector\textless{}int\textgreater{} trackPlanesHitList \\
\textit{List of hit planes, with missed planes excluded from the vector. Ex. 1 2 4 5 8 9} \\ \hline


std::vector\textless{}std::vector\textless{}double\textgreater{} \textgreater{} wireUVPositions \\
\textit{Wire center U and V postions, first vector is track param vector size 5 (0 1 2 unfilled, 3 is U, 4 is V), second vector is planenumber from 0 - 32  (formatted this way to align with other similar vectors - can probably be reduced.)} \\ \hline

std::vector\textless{}double\textgreater{} measuredDCAs \\
\textit{Vector of measured DCAs for hit planes, with size 33. Mainly to hold on to smearing values for now.} \\ \hline

std::vector\textless{}double\textgreater{} uncorrErrors \\
\textit{Size 33, errors filled with digit errors or straw diameter errors - uncorrected.} \\ \hline

std::vector\textless{}double\textgreater{} UVerrors \\
\textit{Size 33, errors corrected each iteration with angular correction from dca/straw error to UV error - errors that are used in the fit.} \\ \hline

std::vector\textless{}gm2strawtracker::GeaneHitSide\textgreater{} inputHitSides \\
\textit{Size 33, LR sides coming from from upstream of the fitting code - and modified within the fitting code depending on the fit mode.}  \\ \hline

std::vector\textless{}double\textgreater{} dcaErrors \\
\textit{Size 33, dca errors directly copied from straw digit art record.} \\ \hline

std::vector\textless{}double\textgreater{} strawDiamErrors \\
\textit{Size 33, vector of straw diameter errors = strawdiameter/$\sqrt{12}$.} \\ \hline

std::vector\textless{}double\textgreater{} planeXPositions \\
\textit{Vector of X postions of hit wire planes with size 33 (0 - 32), 0 plane being in front of the first module that was hit.} \\ \hline

std::vector\textless{}std::vector\textless{}double\textgreater{} \textgreater{} geaneMeasuredParameters \\
\textit{Measured GEANE parameters, first vector is param num 0 - 4, second vector is plane number 0 - 32, units are MeV mm. 1/P, Pu/Pz, Pv/Pz, U, V - only U or V is filled at the start of the GEANE fitting module.} \\ \hline

std::vector\textless{}std::vector\textless{}double\textgreater{} \textgreater{} geanePredictedParameters \\ 
\textit{Predicted GEANE parameters, first vector is param num 0 - 4, second vector is plane number 0 - 32, units are MeV mm. 1/P, Py/Px, Pz/Px, Y, Z - all params filled in tracing stage of GEANE fitting module - coord system has to be orthogonal - converted to UV locally in the fitting module.} \\ \hline

std::vector\textless{}GeaneHitSide\textgreater{} geaneHitSides \\
\textit{Vector of GeaneHitSide enums with size 33 describing the sides of the wires that each hit of the track passed.} \\ \hline
 

std::vector\textless{}Eigen::MatrixXd\textgreater{} geaneTransportMatrices \\
std::vector\textless{}std::vector\textless{}double\textgreater{} \textgreater{} geaneTransportMatricesData \\ 
\textit{Units of GeV cm - transport matrices between planes tracked to in GEANE fitting module, 5x5 objects. Vector has size 33.} \\ \hline

std::vector\textless{}Eigen::MatrixXd\textgreater{} geaneErrorMatrices \\ 
std::vector\textless{}std::vector\textless{}double\textgreater{} \textgreater{} geaneErrorMatricesData \\ 
\textit{Error matrices on tracked to planes, units GeV cm, size 33.} \\ \hline

Eigen::MatrixXd covarianceTotalInverse \\
std::vector\textless{}double\textgreater{} covarianceTotalInverseData \\
\textit{5x5 inverse of total covariance matrix for track. Diagonals represent errors in 5 track paramaters on plane 0.} \\ \hline

std::vector\textless{}Eigen::VectorXd\textgreater{} paramPredictedInUVEigen \\ 
std::vector\textless{}std::vector\textless{}double\textgreater{} \textgreater{} paramPredictedInUVEigenData \\ 
\textit{Predicted parameters in UV space as an eigen object (converted from geanePredictedParameters above) for calculation convenience and some LR information storage. Order of vectors is switched here, first is planenum, second is paramnum, units are MeV mm.} \\ \hline

\textit{Objects below here are full track objects with larger sizes, held on to for fast sequence checking. Units GeV cm.} \\ \hline

std::vector\textless{}Eigen::MatrixXd\textgreater{} extendedTransportMatrixBegToEnd \\
std::vector\textless{}std::vector\textless{}double\textgreater{} \textgreater{} extendedTransportMatrixBegToEndData \\ 
\textit{Accumulated/combined transport matrices from starting plane to all following planes.} \\ \hline

Eigen::MatrixXd extendedCombinedTransportMatricesTranspose \\
std::vector\textless{}double\textgreater{} extendedCombinedTransportMatricesTransposeData \\
\textit{Transpose of larger eigen object composed of above begtoend transport matrices.} \\ \hline

Eigen::MatrixXd extendedReducedMatrix \\
std::vector\textless{}double\textgreater{} extendedReducedMatrixData \\ 
\textit{Total error correlation matrix, reduced to size NxN (N = num planes hit).} \\ \hline

Eigen::MatrixXd extendedReducedMatrixInverse \\ 
std::vector\textless{}double\textgreater{} extendedReducedMatrixInverseData \\
\textit{Inverse of above saved once for LR checking.} \\ \hline

Eigen::MatrixXd extendedModifiedReducedMatrix \\ 
std::vector\textless{}double\textgreater{} extendedModifiedReducedMatrixData \\
\textit{Reduced matrix from above that's going to be modified into a hybrid error matrix separately for U and V fits but that needs to be held onto for all sequences.} \\ \hline


  \hline

% \end{tabular}
\end{longtable}


\section{Plots}
\label{sec:Plots}

It'd probably be good to make one big set of plots to toss into the back for people to look at/reference. All sorts of pulls, parameter plots, residuals, etc. Sort of like Renee's verification pdf.

Also included failed plots and single event viewer plots.



\end{document}
